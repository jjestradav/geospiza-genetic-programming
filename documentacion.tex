\documentclass[10pt,a4paper]{article}
\usepackage[utf8]{inputenc}
\usepackage[spanish]{babel}
\usepackage{textcomp}
\usepackage[left=2cm,right=2cm,top=3cm,bottom=2cm]{geometry}
\author{José Isaac Zeledón Jiménez, Jonathan Estrada Vargas}
\title{Proyecto 0}
\begin{document}
\begin{titlepage}
\begin{center}
\begin{large}
UNIVERSIDAD NACIONAL\\
COSTA RICA \\
\end{large}
\vspace*{1cm}
\begin{large}
Facultad de Ciencias Exactas y Naturales
\end{large} 
\vspace*{1.8cm}\\
Asignatura:\\
\vspace*{2mm}
\begin{large}
Paradigmas de Programación\\
\end{large}
\vspace*{12mm}
\begin{large}
\textbf{PROYECTO 0: 
PROGRAMACIÓN GÉNETICA
}\\
\end{large}
\vspace*{1.8cm}
Profesor:\\
\vspace*{5mm}
\begin{large}
Eddy Miguel Ramírez\\
\end{large}
\vspace*{1.8cm}
Estudiantes: \\
\vspace*{5mm}
\begin{large}
Andrés Jiménez Elizondo\\
Jonathan Estrada Vargas\\
José Isaac Zeledón Jiménez\\
\end{large}
\vspace*{1.8cm}
II CICLO\\
\vspace*{1.8cm}
2019
\end{center}
\end{titlepage}
\tableofcontents
\pagebreak
\section{Introducción}
	Este proyecto que tiene el objetivo de que el equipo de desarrollo del mismo conozca el paradigma de programación funcional, además de conocer un nuevo concepto que son los algoritmos genéticos.\\
	El profesor presentó el problema de encontrar utilizando las técnicas de algoritmos géneticos, las funciones que cumplieran que al ser evaluadas con distintos valores de x y se acercaran lo máximo posible a los z dados en la tripletas de que el profesor nos facilitó.\\
	Las técnicas de algoritmos genéticos utilizadas son las básicas consideradas en este tipo de programación. El problema se enfretó de la manera que se le llamaría generacional con elitismo.\\
	Los 
\section{Descripción del Problema}
 

\section{Especificacion de la solución}
	 
\section{Problemas encontrados}
\section{Conclusiones}

\end{document}
\end{document}
